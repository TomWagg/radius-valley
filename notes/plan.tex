\documentclass[12pt, letterpaper, twoside]{article}
\usepackage{nopageno,epsfig, amsmath, amssymb}
\usepackage{physics}
\usepackage{mathtools}
\usepackage{hyperref}
\usepackage{xcolor}
\hypersetup{
    colorlinks,
    linkcolor={blue},
    citecolor={blue},
    urlcolor={blue}
}
\usepackage{empheq}
\usepackage{wrapfig}

\usepackage[letterpaper,
            margin=0.8in]{geometry}


\title{Radius Valley Project Plan}
\date{}

\newcommand{\todo}[1]{{\color{red}\begin{center}TODO: #1\end{center}}}

% custom function for adding units
\makeatletter
\newcommand{\unit}[1]{%
    \,\mathrm{#1}\checknextarg}
\newcommand{\checknextarg}{\@ifnextchar\bgroup{\gobblenextarg}{}}
\newcommand{\gobblenextarg}[1]{\,\mathrm{#1}\@ifnextchar\bgroup{\gobblenextarg}{}}
\makeatother

\newcommand{\avg}[1]{\left\langle #1 \right\rangle}
\newcommand{\angstrom}{\mbox{\normalfont\AA}}
\allowdisplaybreaks

\begin{document}

\maketitle{}

\vspace{-2.5cm}

\section*{General Concept}
The overall plan is to use multi-transiting systems to break the degeneracies in measuring exoplanet radii. Since each planet orbits the \textit{same} star we can fix several parameters and better fit the rest!

\section*{Radius Fitting Plan}
We'll collect any 3+ multiplicity systems that span the radius valley and for each we aim to derive estimates of their planetary radii using either \texttt{exoplanet} or Julia. In each case we will need to consider the following:

\begin{enumerate}
    \item \textbf{Detrending:} Decide how to convert TPFs to lightcurves
    \begin{itemize}
        \item Correct using pixel level detrending? Does this remove all systematics?
    \end{itemize}
    \item \textbf{TTVs:} For each system we need to make a decision about TTVs (probably using \texttt{TTVFaster} package to check how strong TTVs would be)
    \begin{itemize}
        \item Either we used fixed ephemeris or free transit times
        \item Latter can be too free and smear out the ingress and egress when TTVs are not present
        \item The former is a reasonable assumption if you are not close to resonance or if you're separated by quite a lot (since no TTVs)
    \end{itemize}
    \item \textbf{Limb Darkening:} Choose what order of limb darkening model to use (quadratic default)
    \item \textbf{Sampling:} Ensure time sampling of model doesn't have a strong effect
    \item \textbf{Stellar noise sources}: Consider the presence of starspots and companions
    \begin{itemize}
        \item Starspots (both in transit and out of transit) - out can result in periodic noise, can change mean surface brightness of the star, in of course affects the depth of the transit
        \item Are there stellar companions that result in dilution?
        \item Any outliers from CRs or flares?
    \end{itemize}
    \item \textbf{Model specifics:}
    \begin{itemize}
        \item Does parameterisation of the model affect the results?
        \item Do priors affect the model?
        \item Is the sample properly converged?
    \end{itemize}
\end{enumerate}

\section*{Expected Results}

After working out the radii we can make a statement about the location of the gap. In addition we can work out how that correlates with stellar and orbital parameters.


\end{document}

 